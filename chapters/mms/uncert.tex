% !TeX spellcheck = en_US
\chapter{Models of Uncertainty}
\label{ch:uncertainty}

The \textit{Programming in conditions of uncertainty} deals with decision problems in which the choice is among alternatives whose impact depends not only on the choice of the decision-maker, but also on external factors that cannot be exactly predicted.

These problems are characterized by having:
\begin{itemize}
	\item A preference relation $\Pi$ that is a weak order with a known consistent value function $u(f)$ (replaced by a cost $f$, a single objective)
	
	\item A single decision maker: $|D| = 1 \implies \Pi_d$ reduces to $\Pi$
	
	\item A uncertain environment: $|\Omega| > 1$, the impact of the decision depends also on variables which are not under the control of the decision-maker and about whose value only partial information is available
\end{itemize}

Such problem can be formulated as
$$ \min_{x \in X} f(x, \omega) $$
$$ \omega \in \Omega $$

Where:
\begin{itemize}
	\item $X$ is the set of the alternatives or solutions $x$ 
	
	\item $\Omega$ is the set of the scenarios $\omega$ (in statistics, \textit{sample space})
	
	\item $f(x, \omega)$ is the impact of solution $x$ and scenario $\omega$, $\in \R$
	
	\item Lower impacts are preferable to high ones (or vice versa)
\end{itemize}

The decision maker can choose the alternative $x$, but not the scenario $\omega$, and the choice of $x$ precedes the unraveling of $\gamma$. If $x$ were chosen after $\omega$, the problem would reduce to a Mathematical Programming problem parametrized in $\omega$.

The problems in conditions of uncertainty include two main classes, though other ones have been proposed: 
\begin{itemize}
	\item Decisions in condition of \textit{ignorance}: the only information on scenario $\omega$ is that it falls within $\Omega$
	
	\item Decisions in conditions of \textit{risk}: the probability $\pi_\omega$ of each scenario $\omega \in \Omega$ is known (if $\Omega$ is a discrete set) ot the probability density $\pi (\omega)$ is known (if $\Omega$ is a continuous set)
\end{itemize}

In all these classes, it's possible to define relations of dominance that allow to reduce the set of interesting solutions. In general, these relations will not lead to a single choice, as they typically are partial orders (reflexive, transitive but incomplete).

\section{Dominance relations}
\label{sec:domrel}

\begin{definition}
	We say that alternative $x$ \textbf{strongly dominates} alternative $x'$ when its impact is at least as good in all scenarios $\omega \in \Omega$
	$$ x \wpref{} x' \Leftrightarrow f (x, \omega) \leq f(x', \omega), \quad \forall \omega \in \Omega $$
\end{definition}

Similarly to the concept of Paretian preference, the alternative which are strictly dominated can be rejected a priori. \\

\begin{definition}
	In the finite case, the impact $f (x, \omega)$ can be represented with an \textbf{evaluation matrix} $U$, having the alternatives $x$ on the rows and all the scenarios on the columns.
\end{definition}

In order to find the nondominated alternatives, one must perform pairwise comparisons on the rows of the matrix, exactly as in the Paretian case. \\

\begin{definition}
	We say that alternative $x$ \textbf{probabilistically dominates} alternative $x'$ when for any threshold $\bar f$ the probability that $x$ have impacts not worse than the threshold is not inferior to that of $x'$.
	$$ x \wpref{} x' \Leftrightarrow P \left[f(x, \omega) \leq \bar f\right] \geq P \left[f (x', \omega) \leq \bar f\right], \quad \forall \bar f \in \R $$
\end{definition}

\section{Models of uncertainty}
\label{sec:modelsuncert}

There are two main ways to describe uncertain situations, that reflect into two main forms of the scenario set $\Omega$: 
\begin{itemize}
	\item \textit{Scenario description:} $\Omega$ is a finite set, in which the single scenarios are explicitly listed
	$$ \Omega = \left\{\omega^{(1)}, \dots, \omega^{(|\Omega|)}\right\} $$
	
	\item \textit{Interval description:} $\Omega$ is the Cartesian product of a finite number of real intervals $s$ on the exogenous variables
	$$ \Omega = \left[\omega_1^{\min}, \omega_1^{\max}\right] \times \dots \times \left[\omega_s^{\min}, \omega_s^{\max}\right]$$
\end{itemize}

These are not the only possible cases, $\Omega$ could assume more sophisticated forms. However they are two frequent special cases.

\textit{How to choose?} The appropriate description depends on: 
\begin{itemize}
	\item The precision of the representation
	
	\item The simplicity of the solution process
\end{itemize}

On one hand, the scenario description has a finite number of cases: 
\begin{itemize}
	\item Are they all and only the possible cases? 
	
	\item Are they few or combinatorially many? 
\end{itemize}

On the other hand, the interval description: 
\begin{itemize}
	\item Implies that the exogenous variables are independent (precision)
	
	\item The scenarios are infinitely many, but the worst scenario might be the same for all solutions or easy to find
\end{itemize}

% End L15? This was easy, p247 notes