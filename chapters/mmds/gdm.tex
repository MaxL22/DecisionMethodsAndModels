% !TeX spellcheck = en_US
\chapter{Group decision-making}
\label{ch:gdm}

As in game theory, in group decision-making, the decision-maker set $D$ is finite, but each one does not fix independently the values of the decision variables. The final choice must be taken by the whole set $D$ and all decision-makers must agree before taking it.

We assume: 
\begin{itemize}
	\item A finite feasible region $X$
	
	\item A deterministic environment $|\Omega| = 1 \implies f(x, \bar \omega)$ reduces to $f(x)$
	
	\item An invertible impact function $f$, so that each preference relation $\Pi_d$ can be directly imposed on $X$
	
	\item The decision-makers $d \in D$ are called \textit{individuals}, or \textit{citizens}, \textit{voters}, \textit{agents}, \textit{judges} (the investigations on this problem mainly come from political and social sciences)
	
	\item All preference relations $\Pi_d$ are weak orders; this is important, as it means that each decision-maker is required to respect the strict conditions of rationality
\end{itemize}

These assumptions reduce the complexity to a central point: \textit{how to aggregate the individual preferences $\Pi_d$ into a group preference $\Pi_D$?} After that, the problem becomes equivalent to that of a single decision-maker. \\

\begin{definition}
	Given a set of solutions $X$, we denote as $\D(X)$ the set of all weak orders on the elements of $X$.
\end{definition}

Since every preference relation on solutions is a set of pairs of solutions $\Pi_d \subseteq 2^{X \times X}$, set $\D(X)$ is the collection of all sets that enjoy the reflexive, transitive and complete property, therefore $\D(X) \subseteq 2^{2^{X \times X}}$.

For example, $\D(\{a, b, c\})$ includes 13 relations: 
\begin{itemize}
	\item $6$ total orders corresponding to the permutations
	$$ a \prec b \prec c \quad a \prec c \prec b $$ 
	$$ b \prec a \prec c \quad b \prec c \prec a $$
	$$ c \prec a \prec b \quad c \prec b \prec a $$
	
	\item $6$ weak orders with ties 
	$$ a \prec b \sim c \quad b \prec a \sim c \quad c \prec a \sim b $$
	$$ a \sim b \prec c \quad a \sim c \prec b \quad b \sim c \prec a $$
	
	\item $1$ total indifference relation
	$$ a \sim b \sim c $$
\end{itemize}
Every individual $d \in D$ is associated to one of them.

\section{Social welfare function}

The problem we face is how to derive a group preference relation from a finite set of weak orders, one for each individual. \\

\begin{definition}
	Given a set of solutions $X$, we denote as \textbf{social welfare function} a function that associates to each $|D|$-uple of weak orders on $X$ a "group" weak order on $X$:
	$$ g: \D (X)^{|D|} \rightarrow \D(X) $$
\end{definition}

It's a function that receives the preferences $\Pi_d \in \D(X)$ associated with the individual (all weak orders by assumption) and returns a preference relation $\Pi_D \in 2^{X \times X}$ for the whole group.

We'll discuss some historical proposal and an axiomatic approach, then we'll list the properties that appear to be necessary to characterize an acceptable social welfare function, and see why no function satisfies all properties.

\section{Condorcet method}

The \textit{Condorcet method}, also known as \textit{simple majority method} is based on the following definition: 
$$ x \wpref{D} x' \Leftrightarrow \left|\left\{d \in D \mid x \wpref{d} x' \right\}\right| \geq \left|\left\{d \in D \mid x' \wpref{d} x \right\}\right|$$ 

It performs a sort of "election" on each pair of alternatives: the one preferred by more individuals is preferred by the group. Indifferent individuals have no effect, as they're counted on both sides, only the ones with strict preference matter.

This provides a simple algorithm to compute a social welfare function, however, it has a strong drawback, known as \textit{Condorcet paradox}. The classical example concerns three alternatives $X = \{a, b, c\}$ with
\begin{itemize}
	\item $\Pi_1 = a \prec b \prec c$
	
	\item $\Pi_2 = b \prec c \prec a$
	
	\item $\Pi_3 = c \prec a \prec b$
\end{itemize}

The definition implies that:
$$ a \pref{D} b, \quad b \pref{D} c, \quad c \pref{D} a$$
But then, $\Pi_D$ has a circuit of strict preference: it \textit{does not guarantee that the group preference is a weak order} (not transitive).

This doesn't mean that the method does not work, but it means that it \textit{can fail}. In particular, it can be unable to provide a solution preferable to all other ones, which is the minimum requirement for a choice criterium.

\section{Borda method}

The \textit{Borda method} is based on the auxiliary definition of \textit{Borda count}, building a value function for each individual
$$ B_d (x) = \left|\left\{x' \in X \mid x \wpref{d} x' \right\}\right|$$

Then aggregating them with a simple sum into a group value function
$$ B_D (x) = \sum_{d \in D} B_d (x) $$

The group preference is then derived from the group value function
$$ x \wpref{d} x' \Leftrightarrow B_D (x) \geq B_D (x') $$

The group preference is a weak order by construction and it always allows to build a social welfare function.

A drawback of this method is that the order created depends also on the whole feasible region (the set $X$ appears in the definition), and thus it can be manipulated, affecting the choice.

This allows rank reversal, the preference between two alternatives can depend on other irrelevant ones.

\section{Plurality system}

It's the method traditionally used in elections: a function is built based on the number of individuals that prefer each alternative to all other ones:
$$ V_D (x) = \left|\left\{d \in D \mid x \wpref{d} x' \right\}\right|$$

Then the group preference is derived from the group value function
$$ x \wpref{D} x' \Leftrightarrow V(x) \geq V(x') $$

This is based on a consistent value function, as the Borda method, and therefore gives rise to a weak order. However, it also has the drawback of depending on irrelevant alternatives, even if slightly less than the Borda method, since the value function does not consider all the positions of every solution, but only the winning positions.

Another strong drawback is that this system can select alternatives abhorred by most individuals: it lets compact minorities prevail on disjunted majorities. This derives from the fat that the function $V(x)$ only considers the first position in the individual weak orders, neglecting subsequent positions.

\section{Lexicographic method}

The lexicographic method imposes a total order on the individuals
$$ d_1 \prec \dots \prec d_{|D|} $$
and applies to each pair of alternatives the first strict preference existing:
$$ x \wpref{d} x' \Leftrightarrow \exists d \in D : x \wpref{d} x' \text{ and } x \sim_{d'} x' \ \forall d' < d $$

In words: the individuals are organized into a completely ordered hierarchy, and the first one to not be indifferent on a pair of alternatives decides the order. Two alternatives are indifferent if and only if all individuals consider them as indifferent.

This method describes an extreme version of an absolute monarchy, in which the will of monarch $d_1$ is the law, and if the monarch is indifferent the choice is delegated to another individual, and so on.

This method has several advantages:
\begin{itemize}
	\item It always provides a weak order (most of the time, a total order)
	
	\item It does not suffer from rank reversal
\end{itemize}

Obviously, it's not democratic, therefore: 
\begin{itemize}
	\item Easily unstable, unless the total order is deeply wired in the culture
	
	\item Inefficient, as the people on the lower levels have little incentives to contribute to the group 
\end{itemize}

% End L23, p380 notes

\section{The axiomatic approach}

All historical attempts to build a social welfare function have failed to guarantee all desirable properties. The axiomatic approach: 
\begin{itemize}
	\item Lists the desired properties
	
	\item Tries to design a function that satisfies them by construction
\end{itemize}

In 1950 \href{https://en.wikipedia.org/wiki/Kenneth_Arrow}{\texttt{Kenneth Arrow}} proved that it's impossible to formulate a method enjoying all the properties considered as necessary to aggregate preferences in a democratically acceptable way. 

\begin{definition}
	Given a preference relation $\Pi_d (X) \subseteq X \times X$ and a subset $X' \subseteq X$, the \textbf{restriction} $\Pi_d (X)$ to $X'$ is the preference relation
	$$ \Pi_d (X') = \Pi_d (X) \cap (X' \times X') $$
\end{definition}

Just remove from $\Pi_d (X)$ all pairs $(x,y)$ with $x \notin X'$ or $y \notin X'$. \\

\begin{definition}
	Given a finite set of alternatives $X$ and a finite set $D$ of individuals, a \textbf{preference profile} $\Pi (X) \in \D(X)^{|D|}$ is any vector of $|D|$ weak orders on $X$.
\end{definition}

Its restriction $\Pi (X')$ is the vector composed by the restrictions $\Pi_d (X')$. \\

\begin{definition}
	Given a solution pair $x, y \in X$ and two profiles $\Pi (X)$ and $\Pi'(X)$, we denote that \textbf{$\Pi'$ promotes $x$ over $y$ more than $\Pi$} when:
	\begin{enumerate}
		\item Every individual $d$ who strictly prefers $x$ in $\Pi$ also prefers $x$ in $\Pi'$ 
		$$ x \pref{\Pi, d} y \implies x \pref{\Pi', d} y $$
		
		\item Every individual $d$ who is indifferent between $x$ and $y$ in $\Pi$ weakly prefers $x$ in $\Pi'$
		$$ x \ind{\Pi, d} y \implies x \wpref{\Pi', d} y $$
	\end{enumerate}
\end{definition}
So, the preference for $x$ increases monotonically if $\Pi$ is replaced with $\Pi'$. \\

\begin{definition}
	We denote as \textbf{dictator} an individual $s \in D$ such that
	$$ x \pref{\Pi, s} y \implies x \pref{\Pi, D} y $$
	for every solution pair $x, y \in X$ and for every preference profile $\Pi \in \D^{|D|}$.
\end{definition}

In words, every weak preference of the dictator translates into a weak preference of the group through the social welfare function. \textit{The existence of a dictator is a property of the social welfare function}.

This concept can be relaxed and generalized in two ways. 
\begin{itemize}
	\item Replacing the dictator $s \in D$ with a subset of individuals
	
	\item Restricting the absolute power to a single pair of solutions $(x, y)$ \\
\end{itemize}

\begin{definition}
	Given a solution pair $x,y \in X$, a \textbf{decisive set} for solution pair $(x, y)$ is a subset of individuals $S \subseteq D$ such that
	$$ x \pref{\Pi, d} y, \quad \forall d \in S \implies x \pref{\Pi, D} y $$
\end{definition}

A specific weak unanimous preference of the decisive set translates into a weak preference of the group through the social welfare function. \textit{We could informally name it an oligarchy.}

\subsection{Arrow's axioms}

Kenneth Arrow summarized the desirable properties of a social welfare function into the following set of axioms: 
\begin{enumerate}
	\item \textbf{Nontriviality:} there are at least three alternatives and two individuals
	$$ |X| \geq 3, \quad |D| \geq 2 $$
	\textit{Condorcet method solves satisfactorily the case with two alternatives}
	
	\item \textbf{Universality:} $g (\Pi)$ is defined for all $\Pi \in \D^{|D|}$. The social welfare function must solve the problem for all profiles
	
	\item \textbf{Weak order:} $g(\Pi) \in \D$ for all $\Pi \in \D^{|D|}$. The social welfare function must return a weak order for all profiles
	
	\item \textbf{Independence from irrelevant alternatives:}
	$$ g\left(\Pi(X)\right) = \Pi_D \left(X\right) \implies g \left(\Pi \left(X'\right)\right) = \Pi_D \left(X'\right), \quad \forall X' \subseteq X$$
	The social welfare function on a restricted alternative set is the restriction of the original group preference. \textit{The Borda method and the plurality system violate this axiom}
	
	\item \textbf{Monotony:} given two alternatives $x, y \in X$ and two preference profiles $\Pi (X)$ and $\Pi' (X)$ such that
	\begin{itemize}
		\item $\Pi'(X)$ promotes $x$ over $y$ more that $\Pi (X)$
		
		\item $\Pi(X)$ has the same preferences on all pairs not including $x$
		
		\item $x \pref{\Pi, D} y$
	\end{itemize}
	Then $x \pref{\Pi', D} y$. The social welfare functions maintain a preference for $x$ over $y$ if $x$ is further promoted
	
	\item \textbf{Popular sovereignty:} the co-domain of $g (\cdot)$ id $\D(X)$
	$$ \exists \Pi (X) \in \D (X)^{|D|} : g \left(\Pi \left(X\right)\right) = \bar \Pi, \quad \forall \bar \Pi \in \D \left(X\right) $$
	Every weak order can be obtained choosing $|D|$ suitable preferences. \textit{Function $g(\cdot)$ is surjective}
	
	\item \textbf{Nondictatorship:} no individual is a dictator. \textit{The lexicographic method violates this axiom}
\end{enumerate}

The modern version of Arrow's theorem adopts a weaker axiom to replace monotony (5) and popular sovereignty (6): \textbf{unanimity} (or Pareto efficiency): if all individuals agree on a preference, the group also agrees
$$ x \pref{\Pi, d} y, \ \forall d \in D \implies x \pref{D} y $$

It is weaker because: 
\begin{itemize}
	\item Popular sovereignty guarantees the existence of a profile $\Pi'$ such that $x \pref{\Pi', D} y$
	
	\item The unanimous profile $\Pi$ for which $x \pref{\Pi, d} y$ for all $d \in D$ promotes $x$ over $y$ more than $\Pi'$
	
	\item Monotony guarantees that $x \pref{\Pi', D} y \implies x \pref{\Pi, D} y$ 
\end{itemize}

\begin{theo}[Arrow's theorem]
	Any social welfare function satisfying axioms 1 to 6 implies a dictator.
\end{theo}

The proof goes through the following stages: 
\begin{enumerate}
	\item There is a decisive set for a specific pair of solutions
	
	\item This decisive set can be reduced to a single individual
	
	\item The individual is decisive for every pair of solutions
\end{enumerate}
The steps are based on applying the axioms to special cases and drawing general conclusions.

\subsubsection{Sketch of the proof}

Consider any social welfare function $g(\cdot)$. By \textit{unanimity}, set $D$ is decisive for every pair of alternatives.

Given a specific pair $(w, z)$, is there a smaller decisive set $V = D \setminus \{d\}$? By progressively removing from $V$ individuals that do not affect this, generate a minimal decisive set for $(w, z)$.

Is there a smaller set $V \setminus \{d\}$ that is decisive for another pair $(x,y)$? Remove from $V$ further individuals, possibly changing the pair, until: 
\begin{itemize}
	\item Set $V$ is decisive for $(x,y)$
	
	\item No proper subset of $V$ is decisive for any pair of alternatives
\end{itemize}

By \textit{universality}, function $g(\cdot)$ applies to any preference profile. \textit{So we can challenge it to solve any special case}.

By \textit{nontriviality}, consider a problem with $|X| \geq 3$ and $|D| \geq 2$ and split $D$ into: 
\begin{itemize}
	\item An arbitrary individual $d$ from the decisive set $V$
	
	\item The rest of the decisive set $V \setminus \{d\}$
	
	\item The rest of the decision-maker set $D \setminus V$
\end{itemize}
It is possible that $V$ and $D$ coincide and $D \setminus V$ is empty.

\textit{What about $V \setminus\{d\}$?} If $V \setminus \{d\}$ is empty, $V = \{d\}$ and individual $d$ is decisive for pair $(x, y)$. By contradiction, assume $V \setminus \{d\} \neq \emptyset$ and the following preference profile
$$ 
\begin{array}{c c c}
	\{d\} & V \setminus \{d\} & D \setminus V \\
	\hline
	x & z & y  \\
	y & x & z  \\
	z & y & x  \\
	\dots & \dots & \dots 
\end{array}
$$
where the preference of all individuals on the other alternatives can be ignored by \textit{independence from irrelevant alternatives}.

Notice that: 
\begin{itemize}
	\item $V$ is decisive for $(x,y)$ and $x \pref{d'} y$ for all $d' \in D \implies x \pref{D} y$
	
	\item All members of $V \setminus \{d\}$ think that $z \pref{} y$, while all other individuals think the opposite
	
	\item If we assume that $z \pref{D} y$ then $V \setminus \{d\}$ is decisive for $(y, z)$
	
	\item Since $V$ is a minimal decisive set, this is not possible $\implies y \wpref{D} z$
	
	\item By \textit{weak order}, $x \pref{D} y$ and $y \wpref{D} z \implies x \pref{D} z$
	
	\item This is also impossible, because $d$ would be decisive for $(x, z)$, as the only individual preferring $x$ to $z$
\end{itemize}

So, we go back and conclude that $V = \{d\}$ and $d$ is decisive for $(x, y)$.

Now apply function $g (\cdot)$ to another preference profile 
$$
\begin{array}{c c}
	\{d\} & D \setminus \{d\} \\
	\hline
	x & y \\
	y & w \\
	w & x \\
	\dots & \dots
\end{array}
$$

Notice that: 
\begin{itemize}
	\item By \textit{unanimity}, $y \pref{d'} w$ for all $d' \in D \implies y \pref{D} w$
	
	\item Since $d$ is decisive for $(x,y)$ and $w \pref{d} y \implies x \pref{D} y$, even if all other individuals disagree
	
	\item By \textit{weak order}, $x \pref{D} y$ and $y \pref{D} w \implies x \pref{D} w$
\end{itemize}

But $d$ is the only individual preferring $x$ to $w$: $d$ is decisive for $(x, w)$ for all $w \in X \setminus \{x\}$.

Now apply function $g (\cdot)$ to another (second time) preference profile 
$$
\begin{array}{c c}
	\{d\} & D \setminus \{d\} \\
	\hline
	z & w \\
	x & z \\
	w & x \\
	\dots & \dots
\end{array}
$$
with $w$ and $z$ generic solutions different from $x$ (but possibly equal to $y$).

Notice that: 
\begin{itemize}
	\item By \textit{unanimity}, $z \pref{d'} x$ for all $d' \in D \implies z \pref{D} x$
	
	\item Since $d$ is decisive for $(x,w)$ and $x \pref{d} w \implies x \pref{D} w$, even if all other individuals disagree
	
	\item By \textit{weak order}, $z \pref{D} x$ and $x \pref{D} w \implies z \pref{D} w$
\end{itemize}

But $d$ is the only individual preferring $z$ to $w$: $d$ is decisive for $(z, w)$ for all $w, z \in X \setminus \{x\}$.

Now apply function $g (\cdot)$ to a last preference profile 
$$
\begin{array}{c c}
	\{d\} & D \setminus \{d\} \\
	\hline
	z & w \\
	w & x \\
	x & z \\
	\dots & \dots
\end{array}
$$
with $w$ and $z$ generic solutions different from $x$ (but possibly equal to $y$).

Notice that: 
\begin{itemize}
	\item By \textit{unanimity}, $w \pref{d'} x$ for all $d' \in D \implies w \pref{D} x$
	
	\item Since $d$ is decisive for $(z,w)$ and $z \pref{d} w \implies z \pref{D} w$, even if all other individuals disagree
	
	\item By \textit{weak order}, $z \pref{D} w$ and $w \pref{D} x \implies z \pref{D} x$
\end{itemize}

But $d$ is the only individual preferring $z$ to $x$: $d$ is decisive for $(z, x)$ for all $z \in X \setminus \{x\}$.

In summary, $d$ is decisive for all pairs of alternatives: \textit{$d$ is a dictator.}

\section{Criticisms to Arrow's axioms}

Most of Arrow's axioms have been criticized in the following decades. In particular, the reviewers tried to undermine the proof suggesting that the axioms are not as natural and obvious as they look. The criticisms are partially justified in stating that Arrow's model does not correspond exactly to the real world, but cannot anyway destroy the fundamental core of the theorem, that is the fact that no preference aggregation rule is able to satisfy the requirements that it would be desirable to enjoy.

\subsubsection{Nontriviality}

Of course we need $|D| \geq 2$, but what if $|X| = 2$? \href{https://en.wikipedia.org/wiki/Duncan\_Black}{\texttt{Duncan Black}} showed that if there are only two alternatives, the Condorcet method satisfies all other axioms. 

But how to reduce the alternatives to two? Bipartitism? Primaries? \textit{This just pushes the problem to the lower level.}

\subsubsection{Universality}

Do we really need $g(\cdot)$ to work in all cases? Black proved that the Condorcet method satisfies all axioms when $X$ admits a linear order for which all preferences $\Pi_d$ are "unimodal": following the order, the alternatives first progressively improve, then progressively worsen.

Moreover, sorting the individuals based on their best alternative, the best alternative of the median individual is the best of the group.

\subsubsection{Weak order}

Can we accept incompleteness and/or intransitivity? As already observed, Condorcet circuits open the way to manipulations.

History proposes several examples of parliamentary votes in which three versions of a law had the support of voter groups structured as in the Condorcet paradox, and, in some cases, the \textit{impasse}, was solved by setting an arbitrary order of business to discuss the conflicts between solutions. Different orders of business can lead to different choices and setting the agenda often grants a strong position in parliamentary proceedings.

\subsubsection{Independence from irrelevant alternatives}

Can we really ignore the other alternatives? It depends on the model: the position of two alternatives in the overall order might be related to their appeal to the individual.

A quantitative measure of utility could solve the problem, but that would reintroduce the complexity of multiple-attribute utility theory.

\subsubsection{Unanimity}

Should a unanimous preference always be applied? In the radical democracy modeled by Arrow, it should. Many political ideologies exclude some decisions from the formal aggregation of the individual preferences
\begin{itemize}
	\item "Nonnegotiable values" for the Catholic church
	
	\item Human rights for the liberal thought
	
	\item \dots
\end{itemize}

Of course, this opens the problem of how to define such decisions.

% End L24, finally