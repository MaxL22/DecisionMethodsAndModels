% !TeX spellcheck = en_US
\section{Basic Decision Models}

\subsection{Structured preferences}

\subsubsection{Dominance relation}

Some ways to sort \textit{the stuff}.

\paragraph{Lexicographic order} Order the alternatives w.r.t. the value of the first indicator (for some ordering of the indicators) and break ties with the subsequent one. This yields a total order.

The variant with aspiration levels introduces a "minimum requirement" $\epsilon_i$, rejecting all alternatives with indicator $f_i$ worse than $\epsilon_i$ (higher or lower depending on whether it's a benefit or cost).

\paragraph{Utopia point} Identify an ideal impact with the best possible value for each indicator (optimize them independently) and evaluate all alternatives with the distance from the ideal impact. 

Different definitions of distance yield different results. Some definitions:
\begin{itemize}
	\item $L_1$ Manhattan distance
	$$ d(f, f') = \sum_{l \in P} |f_l - f_l'|$$
	
	\item $L_2$ Euclidean distance
	$$ d(f, f') = \sqrt{\sum_{l \in P} \left(f_l - f_l'\right)^2} $$
	
	\item $L_\infty$ Chebyshev distance/maximum norm
	$$ d(f,f') = \max_{l \in P} |f_l - f_l'| $$
\end{itemize}

\paragraph{Borda count} In the case of finite alternatives, they can be sorted by counting how many alternatives are worse than each one
$$ B(f) = \left|\left\{f' \in F \mid f \wpref{} f' \right\}\right| $$

\subsection{MAUT}

\subsubsection{Indifference curves}

An \textit{indifference curve} is a subset of the impact space $I \subseteq F$ of reciprocally indifferent impacts. By definition: 
\begin{itemize}
	\item The curves cover $F$
	
	\item Any two curves have empty intersection
	
	\item Weak order on impacts maps to total order on curves
\end{itemize}

Usually, continuity is assumed (they are mathematical objects and not a general set of points), and each indifference curve is expressed in the implicit form $ u(f) = c$, each $c$ identifies a curve.

%When they can be turned in explicit form
%$$ f_l = f_l (c, \dots, f_{l-1}, f_{l+1}, \dots, f_p) $$
%Then the indifference curve is a $p-1$-dimensional hypersurface in the space of indicators $\R^p$.

\subsubsection{MRS}

The Marginal Rate of Substitution $\lambda_{12}$ between two indicators $f_1$ and $f_2$ represents how much of $f_1$ are we willing to "give up" for a unit of $f_2$; e.g., if we're willing to give $4$ units of $f_1$ for a unit of $f_2$ then $\lambda_{12} = 1/4$.

It's the ratio of the partial derivatives of the utility function w.r.t. $f_1$ and $f_2$
$$ \lambda_{12} (f) = \frac{ \frac{\partial u }{\partial f_1} }{ \frac{\partial u}{\partial f_2} } $$

A uniform MRS corresponds to a linear utility function $u(f) = w_1 f_1 + w_2 f_2$, and as such:
$$ \lambda_{12} (f) = \frac{w_1}{w_2} $$

It represents the steepness of the indifference curve (slope).

\subsection{Mathematical Programming: \textit{Amateur hour}}

The general process for solving MP problems is: 
\begin{enumerate}
	\item Draw a graphical representation of the feasible region
	
	\item Find which points are in the candidate set
	
	\item Write the generalized Lagrangian function
	
	\item Write the KKT conditions
	
	\item Solve the system of conditions to reject candidate points, hoping that few remain (add nonregular points after this)
	
	\item Evaluate the function in all the remaining points, choose the optimum
\end{enumerate}

\textit{Easy enough right? (It's not)}