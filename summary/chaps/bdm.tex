% !TeX spellcheck = en_US
\section{Basic Decision Models}

\subsection{Dominance relation}

A solution \textbf{dominates} another when the impact of the former is preferable to the impact of the latter. When solving a decision problem, it looks natural to choose among the nondominated solutions. 

On a strict preference graph, the nondominated solutions correspond to nodes with no ingoing arcs. 

\subsubsection{Decision making on weak orders}

If a preference relation is a weak order
\begin{itemize}
	\item the induced dominance is a weak order
	
	\item and the solution set is finite and nonempty, nondominated solutions exists and are all mutually indifferent
\end{itemize}

A \textbf{value function} $f: F \rightarrow \R$ associates each impact to a real value. A value function is \textbf{consistent} with preference relation $\Pi$ when 
$$ f \preceq f' \Leftrightarrow v(f) \geq v(f'), \quad \forall f,f' \in F$$

If a preference relation $\Pi$ admits a consistent value function, then $\Pi$ it's a weak order. The opposite is not always true. Consistent value functions are not unique, each one admits infinite equivalent ones that sort impacts in the same way, with different values (composition with any strictly increasing function).

\subsubsection{Weak order preference models}

Some ways to order preferences starting from a weak order.

\paragraph{Lexicographic order} Order the alternatives w.r.t. the value of the first indicator (for some defined ordering of the indicators) and break ties with the subsequent one. This yields a total order.

The variant \textbf{with aspiration levels} introduces a "minimum requirement" $\epsilon_l$, rejecting all alternatives with indicator $f_l$ worse than $\epsilon_l$ (higher or lower depending on whether it's a benefit or cost). It then applies the lexicographic order on the restricted feasible region.

\paragraph{Utopia point} Identify an ideal impact, with the best possible value for each indicator (optimize them independently or estimate such point) and evaluate all alternatives with the distance from the ideal impact. 

Different definitions of distance yield different results. Some definitions:
\begin{itemize}
	\item $L_1$ Manhattan distance
	$$ d(f, f') = \sum_{l \in P} |f_l - f_l'|$$
	
	\item $L_2$ Euclidean distance
	$$ d(f, f') = \sqrt{\sum_{l \in P} \left(f_l - f_l'\right)^2} $$
	
	\item $L_\infty$ Chebyshev distance/maximum norm
	$$ d(f,f') = \max_{l \in P} |f_l - f_l'| $$
\end{itemize}

\paragraph{Borda count} In the case of finite alternatives, they can be sorted by counting how many alternatives are worse than each one
$$ B(f) = \left|\left\{f' \in F \mid f \wpref{} f' \right\}\right| $$
Then 
$$ f \wpref{} f' \Leftrightarrow B(f) > B(f') $$ 

\subsection{MAUT}

Multi Attribute Utility theory assumes the preference relation of the decision-maker is a weak order, admitting a consistent value function. We need to derive such value function from the preference relation.

\subsubsection{Indifference curves}

An \textbf{indifference curve} is a subset of the impact space $I \subseteq F$ of reciprocally indifferent impacts. By definition: 
\begin{itemize}
	\item The curves cover $F$
	
	\item Any two curves have empty intersection
	
	\item Weak order on impacts maps to total order on curves
\end{itemize}

Usually, \textbf{continuity} is assumed (they are mathematical objects and not a general set of points), and each indifference curve is expressed in the implicit form $ u(f) = c$, each $c$ identifies a curve.

%When they can be turned in explicit form
%$$ f_l = f_l (c, \dots, f_{l-1}, f_{l+1}, \dots, f_p) $$
%Then the indifference curve is a $p-1$-dimensional hypersurface in the space of indicators $\R^p$.

\subsubsection{Determining the utility function}

The general process requires sampling and guessing a utility function, with consistency checks. This is complex and error prone. 

Some properties which allow for easier estimates of $u(f)$: 
\begin{itemize}
	\item \textbf{Invertibility:} $u(f)$ can be solved w.r.t. each $f_l$
	
	\item \textbf{Monotony:} to compensate the variation of an indicator, the others must vary in a well-defined direction
	
	\item \textbf{Convexity} or \textbf{concavity:} the indifference curves compensate for the increase 	of an indicator by a certain amount with variations of the other ones that increase (or decrease) with the value of the first indicator
\end{itemize}

\subsubsection{Additive utility function}

A utility function is additive when it can be expressed as a sum of functions of the single indicators. If the utility function is additive, the problem to estimate it can be reduced to the estimation of $p$ single-variable functions, simplifying the process.

\paragraph{Preferential independence} A subset of indicators $L$ is \textbf{preferentially independent} from the complementary subset $P \setminus L$ when given two impacts with identical values of $P \setminus L$ the preference relation does not depend on values in $P \setminus L$.

\paragraph{Mutual preferential independence} A problem enjoys \textbf{mutual preferential independence} when every proper subset of indicator is independent from its complement. 

Mutual preferential independence is \textit{necessary for additivity}. Intuitively if subsets of indicators depend on each other there can't be a way to express the utility function in an additive way by summing such indicators (or value of the subsets).

For a decision problem with $p \geq 3$ indicators mutual preferential independence is a sufficient condition for additivity.

\subsubsection{MRS}

The \textbf{Marginal Rate of Substitution} $\lambda_{12}$ between two indicators $f_1$ and $f_2$ represents how much of $f_1$ are we willing to \textit{"give up"} for a unit of $f_2$; e.g., if we're willing to give $4$ units of $f_1$ for a unit of $f_2$ then $\lambda_{12} = 1/4$.

It's the ratio of the partial derivatives of the utility function with respect to $f_1$ and $f_2$
$$ \lambda_{12} (f) = \frac{ \frac{\partial u }{\partial f_1} }{ \frac{\partial u}{\partial f_2} } $$

A uniform MRS corresponds to a linear utility function $u(f) = w_1 f_1 + w_2 f_2$, and as such:
$$ \lambda_{12} (f) = \frac{w_1}{w_2} $$

It represents the steepness of the indifference curve (slope).

\paragraph{Corresponding trade-off condition} We denote as \textbf{corresponding trade-off condition the property}
$$ \lambda_{12} (f_1', f_2') \lambda_{12} (f_1'', f_2'') = \lambda_{12} (f_1'', f_2') \lambda_{12} (f_1', f_2'')$$

Multiplying two MRSs on a diagonal equate to multiplying the values on the opposite diagonal.

A preference relation $\Pi$ admits an additive utility function if and only if it enjoys both mutual preferential independence and the corresponding trade-off condition.

\subsubsection{Building an additive utility function}

Usually a utility function uses normalized pure values for utilities (pure numbers $\in [0,1]$) and combines them with nonnegative weights whose total sum is 1. 

Then we need to build a function for each one-dimensional impact, for example through the bisection method: get the decision maker to point extreme impacts and points in the middle of each "segment" created, until a function can be determined.

\paragraph{Determining the weights} Identify a sufficient number of indifferent impacts and solve the resulting constraints on the weighs (linear equation system).

If indifference is imprecise, this can give wrong weights, so build a complete pairwise comparison for all indicators. The \textbf{pairwise comparison matrix} is the matrix reporting all rates of substitution between normalized utilities, expressing relative weight. 

A correct pairwise comparison matrix enjoys: 
\begin{itemize}
	\item \textbf{Positivity:} each entry is positive, $\tilde \lambda_{lm} > 0$, $\forall l,m \in P$
	
	\item \textbf{Reciprocity:} $\tilde \lambda_{lm} = 1/ \tilde \lambda_{ml}$, $\forall l,m \in P$
	
	\item \textbf{Consistency:} $\tilde \lambda_{ln} = \tilde \lambda_{lm} \cdot \tilde \lambda_{mn}$, $\forall l,m,n \in P$
\end{itemize}

To determine the weights from (correct) a matrix: 
\begin{itemize}
	\item Calculate the total of each column
	
	\item Divide each entry in the column for the total
	
	\item For each row calculate the mean of the resulting values, those means compose the weight vector
\end{itemize}

\subsection{Mathematical Programming: \textit{How to}}

Mathematical programming assumes decision problems with a preference relation which admits a known consistent utility function, reducing the problem to classical optimization. 

The general process for solving MP problems is: 
\begin{enumerate}
	\item Draw a graphical representation of the feasible region
	
	\item Find nonregular points
	
	\item Write the generalized Lagrangian function
	
	\item Write the KKT conditions
	
	\item Solve the system of conditions to reject candidate points, hoping that few remain (add nonregular points after this)
	
	\item Evaluate the function in all the remaining points, choosing the optimum
\end{enumerate}

\textit{Easy enough right? (It's not)}

\textit{Let's use an example:
\begin{align*}
	\min f(x) & = (x_1 - 1)^2 + x_2^2 \\
	g_1 (x) & = -x_1^2 -x_2^2 + 4 \leq 0 \\
	g_2 (x) & = x_1 - 3/2 \leq 0
\end{align*}}

\paragraph{Nonregular points} All points in which the gradients of the active constraints are linearly independent are regular. Only active constraints must be considered, the whole feasible region is composed of regular points. A constraint is active when $= 0$. 

\textit{Why are active constraints zero? The constraints are active only on the borders of the feasible region, since a optimal solution can only be found on that border, "pushing the boundaries" of the problem. Each point strictly inside the feasible region is next to another point, slightly better, slightly more towards the border.}

Calculate the gradients of each constraint and check wether they can be zero or not. If the gradient can be zero and in such point: 
\begin{itemize}
	\item The constraint is inactive: business as usual
	
	\item The constraint is active: nonregular point, has to be added to the candidate set
\end{itemize}

\textit{In our example, the gradients are: 
\begin{align*}
	\nabla g_1 (x) & = \left[ - 2 x_1 \  - 2 x_2 \right] \\
	\nabla g_2 (x) & = \left[ 1 \ 0 \right]
\end{align*}
And:}
\begin{itemize}
	\item \textit{The first one is 0 only in the origin, point in which the constraint is nonactive ($ g_1 (0,0) = 4 $)}
	
	\item \textit{The second one is never zero}
\end{itemize}

% TODO: equality constraints? When to add? 
Then check points in which pairs of constraints are active, i.e., make a system in which both are zero (I think equality constraints always have to be added? I'll get back to you on that, not sure).

\textit{In our example:
$$
\begin{cases}
	-x_1^2 - x_2^2 + 4 = 0 \\
	x_1 - 3/2 = 0 
\end{cases}
\implies
\begin{cases}
	x_1 = 3/2 \\
	x_2^2 = 7/4
\end{cases}
$$
From which we get the points
$$ A = \left(\frac{3}{2}, \frac{\sqrt{7}}{2}\right), \quad B \left(\frac{3}{2}, - \frac{\sqrt{7}}{2}\right) $$}

Check wether the gradients are linearly independent or not in the points found. To verify this the simplest way is to compose a $2 \times 2$ matrix with the values of the gradients considered in each point, if the determinant of such matrix is nonzero the gradients are linearly independent in the point considered.

\textit{In our example, $\nabla g_1 (A) = \left[- 3 \ -\sqrt{7}\right]$ and $\nabla g_2 (A) = \left[1 \ 0 \right]$, the resulting matrix being
$$
M = \left[\begin{array}{c c}
	-3 & 1 \\ -\sqrt{7} & 0
\end{array} \right]
$$
Whose determinant is: 
$$ det(M) = (-3 \cdot 0) - (-\sqrt{7} \cdot 1) = \sqrt{7} \neq 0$$
So the gradients are linearly independent.}

Then check points in which triples of constraints are active, similarly to earlier. You can guess how this goes on.

\textit{In our example there are no more constraints, but you would simply check that the gradients are linearly independent in the points resulting from the system given by $g_1(x) = 0$, $g_2(x) = 0$, $g_3(x) = 0$.}

The aim of this phase is only to find nonregular points. Points with not linearly independent gradients are nonregular and as such have to be added to the candidate set after "sifting" with the KKT conditions.

\paragraph{Generalized Lagrangian function} The generalized Lagrangian function is defined as:
$$ \ell (x) = f(x) + \sum_{i = 1}^s \lambda_i h_i (x) + \sum_{j = 1}^m \mu_j g_j (x) $$
With $\lambda_i$ free multipliers and $h_i (x)$ equality constraints (always active).

\textit{In our example, the function becomes:
\begin{align*}
	\ell (x) & = f(x) + \mu_1 g_1 (x) + \mu_2 g_2 (x) \\
	& = (x_1 - 1)^2 + x_2^2 + \mu_1\left(-x_1^2 - x_2^2 + 4\right) + \mu_2 (x_1 - 3/2)
\end{align*}
}

\paragraph{KKT Conditions} The KKT conditions state that if a point is regular and locally minimal: 
\begin{enumerate}
	\item \textbf{Stationarity:} the partial derivatives of the Lagrangian function w.r.t. the $x$ variables are equal to zero ($\partial \ell / \partial x_i$ = 0)
	
	\item The partial derivatives of the Lagrangian function w.r.t. the $\lambda$ multipliers are equal to zero ($\partial \ell / \partial \lambda_i = h_j = 0$), that is, the equality constraints are respected (just add the equality constraints)
	
	\item \textbf{Dual feasibility:} the product of the functions expressing the inequality constraints, times the corresponding multipliers are equal to zero ($\mu_k g_k = 0$)
	
	\item \textbf{Primal feasibility:} all inequalities constraints are satisfied
	
	\item \textbf{Complementary slackness:} all multipliers of the inequality constraints are nonnegative ($\mu_k \geq 0$)
\end{enumerate}
These conditions allow to restrict the number of candidate points from \textit{all regular and nonregular points} to \textit{nonregular and a few regular}.

\textit{In our example, the conditions become}
\begin{align*}
	\partial \ell / \partial x_1 & = 2(x_1 - 1) - 2 \mu_1 x_1 + \mu_2 = 0 \\
	\partial \ell / \partial x_2 & = 2x_2 - 2 \mu_1 x_2 = 0 \\
	\mu_1 g_1 & = \mu_1(-x_1^2 -x_2^2 + 4) = 0 \\
	\mu_2 g_2 & = \mu_2 (x_1 - 3/2) = 0 \\
	g_1 & = -x_1^2 -x_2^2 + 4 \leq 0 \\
	g_2 & = x_1 - 3/2 \leq 0 \\
	\mu_1 & \geq 0 \\
	\mu_2 & \geq 0
\end{align*}
\textit{No equalities here, but just add them as constraint and remember to find the value of the free multiplier $\lambda$.}

\paragraph{Solving conditions} To solve the system without exhaustively exploring all possible cases one can use a search tree, whose nodes divide the feasible region in disjoint parts. To do that we build on the stricter conditions, i.e., the complementary slackness conditions (the products $\mu_k g_k = 0$, start with the simplest one). Given a constraint, we can distinguish two cases:
\begin{enumerate}
	\item $\mu_k = 0$ and $g_k \leq 0$
	
	\item $\mu_k > 0$ and $g_k = 0$
\end{enumerate}

These assumption simplify the system, allowing it to be analyzed in an easier way. The process can be repeated on the resulting sub-problems, if necessary.

The idea is to restrict the possible solutions and find them a little at a time. Divide in "easy" sub-problems and all solutions to such problems are candidate points.

\textit{In our example, the constraint chosen is $\mu_2 g_2 = 0$. We now consider the case $P^1$, with $\mu_2 > 0$ and $g_2 (x) = 0$, and thus $x_1 = 3/2$. The constraints now become:
\begin{align*}
	1 - 3 \mu_1 + \mu_2 & = 0 \\
	x_2 (1 - \mu_1) & = 0 \\
	\mu_1 (7/4 - x_2^2) & = 0 \\
	7/4 - x_2^2 & = 0 \\
	\mu_1 & \geq 0
\end{align*}
We can now say that $\mu_1 = (\mu_2 + 1)/3 > 0$ and $x_2^2 = 7/4$. This yields candidate points:
$$ A = \left(\frac{3}{2}, \frac{\sqrt{7}}{2}\right), \quad B \left(\frac{3}{2}, - \frac{\sqrt{7}}{2}\right) $$
}

\textit{Now consider $P^2$, with $\mu_2 = 0$ and $g_2 \leq 0$, the constraints become:
\begin{align*}
	2(x_1 - 1) - 2 \mu_1 x_1 & = 0 & \implies x_1 (1 - \mu_1) = 1 & \implies \mu_1 \neq 1 \\
	x_2 (1 - \mu_1) & = 0 & \implies x_2 = 0 \\
	\mu_1(-x_1^2 -x_2^2 + 4) & = 0 & \implies \mu_1 (4 - x_1^2) = 0 \\
	x_1^2 + x_2^2 & \geq 4 & \implies x_1^2 \geq 4 \\
	x_1 & \leq 3/2 \\
	\mu_1 & \geq 0 \\
	& \implies x_1 = -2
\end{align*}
This yields candidate point:
$$ C = \left(-2, 0\right)$$}

\paragraph{Choose optimum} We now want to evaluate the function in all the candidate points (remember to consider nonregular points) and choose the best solution.

\textit{In our example:
$$
\begin{cases}
	f(A) = 2 \\
	f(B) = 2 \\
	f(C) = 9
\end{cases}
$$
Which implies that both $A$ and $B$ are globally optimal points.}