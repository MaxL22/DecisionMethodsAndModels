% !TeX spellcheck = en_US
\section{Decision Problems}

\subsection{Fundamental definitions}

A decision problem is defined by a 6-uple: 
$$ P = (X, \Omega, F, f, D, \Pi)$$
Where
\begin{itemize}
	\item $X$ is the feasible region, set of all alternatives
	
	\item $\Omega$ is the sample space, set of all possible scenarios
	
	\item $F$ is the indicator space, set of all possible impacts 
	
	\item $f: X \times \Omega \rightarrow F$ is the impact function, associates each configuration of the system to a impact
	
	\item $D$ is the set of all decision-makers
	
	\item $\Pi: D \rightarrow 2^{F \times F}$ is the preference function, which associates to each decision maker a subset of impact pairs
\end{itemize}

The \textbf{alternatives} formally describe the events under the control of the decision-makers. $X$ includes all possible choices.

The \textbf{scenarios} formally describe the events out of the control of the decision-makers. $\Omega$ defines all such events.

The \textbf{impacts} model all aspects relevant to the decision, they are described quantitatively as vector of real numbers.

The \textbf{impact function} is a vectorial function which associates each configuration to an impact.

A \textbf{decision-maker} is whoever takes part in the decision. 

The decision-maker takes part directly in the choice of alternative, holding the power and responsibility to decide, while \textbf{stakeholders} are all the subjects who don't actively participate in the decision but whose interests are affected by its outcome.

\subsection{Preference Relation}

For each decision maker, we need a relation between pair of impacts to determine which ones are preferable. $\Pi: D \rightarrow 2^{F \times F}$ associates each decision-maker to a subset of impact pairs.

A \textbf{weak preference} $f \wpref{d} f'$ is when the decision-maker accepts the exchange of $f$ for $f'$.

Two impacts are \textbf{indifferent} if the decision-maker considers both equally as satisfactory, i.e., is willing to exchange one for the other in both directions.

Two impacts are \textbf{incomparable} if the decision-maker is unable or unwilling to choose between them, rejecting the exchange in both directions.

In a decision problem, an \textbf{outperformer} $x \in X$ for $d \in D$ is an alternative which performs at least as good as any other $x' \in X$ in every scenario $\omega \in \Omega$
$$ (f(x, \omega), f(x', \omega)) \in \Pi_d, \quad \forall \omega \in \Omega, x' \in X $$
and is strictly better for at least one comparison/scenario
$$ \exists x^\circ \in X, \omega^\circ \in \Omega \quad (f(x, \omega), f(x^\circ, \omega)) \in \Pi_d \wedge (f(x^\circ), \omega), f(x, \omega)) \notin \Pi$$

\subsubsection{Properties}

Considering impact set $F$ and a binary relation $\Pi$, the main properties of said relation can be:
\begin{itemize}
	\item \textbf{Reflexivity:} $\forall a \in F$, $(a,a) \in \Pi$; to check for it, on the graph representation every impact has to have a self-loop
	
	\item \textbf{Transitivity:} $\forall a,b,c \in F$, if $(a,b) \in \Pi$ and $(b,c) \in \Pi$ then $(a,c) \in \Pi$; to check for it, on the graph representation follow every arc and check that every possible "triangle" is complete
	
	\item \textbf{Antisymmetry:} $\forall a,b \in F$, if $(a,b) \in \Pi$ and $(b,a) \in \Pi$ then $a = b$; to check for it, on the graph representation there must be no impacts that point at each other
	
	\item \textbf{Completeness:} $\forall a,b \in F$, if $(a,b) \notin \Pi$ then $(b, a) \in \Pi$; to check for it, on the graph representation, from each node there must be an arc (either outgoing or incoming) connecting it to every other
\end{itemize}

\subsubsection{Types of relations (orders)}

Combining the properties, we can get different kinds of preferences: 
\begin{itemize}
	\item \textbf{Preorder:} reflexivity and transitivity. Guarantees that the set of nondominated alternatives is nonempty, even if not all alternatives are comparable, we can find at least one nondominated solution
	
	\item \textbf{Partial order:} reflexivity, transitivity and antisymmetry. This limits indifference, but still allows for incomparability, and thus not always leading to a definitive choice
	
	\item \textbf{Weak order:} reflexivity, transitivity and completeness. Guarantees that nondominated solutions exists and are all mutually indifferent, any of them can be chosen. Such orders admit representation by a value function (with ties), turning the decision problem into an optimization problem
	
	\item \textbf{Total order:} reflexivity, transitivity, antisymmetry and completeness. Provides a unique linear ranking of alternatives, there is always a unique best alternative
\end{itemize}

\subsubsection{Derived relations}

From the weak preference relation, one can derive: 
\begin{itemize}
	\item \textbf{Indifference relation:} $\indiff{\Pi}$
	$$ (a,b),(b,a) \in \indiff{\Pi} \Leftrightarrow (a,b) \in \Pi \wedge (b,a) \in \Pi $$
	To build it from the graph representation, add all self loops and each pair of arcs that point at each other
	
	\item \textbf{Strict preference relation:} $\spref{\Pi}$
	$$ (a,b) \in \spref{\Pi} \Leftrightarrow (a,b) \in \Pi \wedge (b,a) \notin \Pi $$
	To build it from the graph representation, add all arcs which do not have an equal one in the opposite direction (the set difference of the last one w.r.t. $\Pi$)
	
	\item \textbf{Incomparability relation:} $\inc{\Pi}$
	$$ (a,b), (b,a) \in \inc{\Pi} \Leftrightarrow (a,b) \notin \Pi \wedge (b,a) \notin \Pi $$
	To build it from the graph representation, add all arcs which are not present in either direction in the graph
\end{itemize}