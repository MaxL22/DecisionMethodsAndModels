% !TeX spellcheck = en_US
\section{Models with complex preferences}

The preference relation now is not a weak order. 

We denote as \textbf{Paretian preference} (in the case of costs) the relation 
$$ \Pi = \left\{(f, f') \mid f_l \leq f_l', \text{ for each } l \in \{1, \dots, p\}\right\} $$
Which is a \textbf{partial order}.

We denote as \textbf{dominated solution} a solution which is not better than w.r.t. each indicator of another solution, and strictly worse in at least one indicator.

We denote as \textbf{Paretian solution} every solution such that no other solution dominates it. We call \textbf{Paretian region} the set of all Paretian solutions.

Paretian solutions are not preferable to all other solutions and are not all reciprocally indifferent. Thus we want to identify the whole Paretian region

\subsection{Identifying the Paretian region}

\subsubsection{Applying the definition}

In the finite case, the Paretian region can be found by applying the definition through pairwise comparisons. 

This is exact but slow.

\subsubsection{Inverse transformation method}

If the solution can be graphically represented, compute the inverse function $\phi: F \rightarrow X$ of $f: X \rightarrow F$, build the image of $X$ in $F$ through $f$, find graphically the nondominated impacts (lower left quadrant empty), find a parametric way to describe such impacts, transform them back using the inverse function. 

This is exact, but human intervention is required and is limited to 2 indicators (\textit{maybe} 3).

\subsubsection{KKT conditions}

KKT conditions can be extended to Paretian preference, by repeating the derivation with minor changes, obtaining a set that is usually larger than the Paretian region (finds an overestimate).

Not usable in discrete problems, as always.  

\subsubsection{Weighted sum method}

Consists in building a linear combination of the indicators and optimizing it. The result is sufficient conditions for a point to be Paretian (underestimate of the region).

% TODO Explain how to actually do an exercise, there is parametric stuff to do

\subsubsection{$\epsilon$-constraint method}

Replace all indicators but one with constraints that require the solution to respect a quality threshold and solve the auxiliary problem. The result is a necessary condition for a point to be Paretian.

It's needed to find parametrically how the Paretian region is described w.r.t. the variation of $\epsilon$. It's just a constraint, so: when it varies, which solutions are feasible? Get the intervals for which the Paretian region doesn't change and optimize the non-replaced indicator for the feasible solutions.

This can be applied to any problem and provides an overestimate of the Paretian region, but it requires to consider all possible values of $\epsilon$ and find all globally optimal solutions, increasing the complexity of the problem. 

\subsection{Weak rationality methods}

Decision-makers often can't estimate correctly, so let's just \textit{embrace} that the pairwise comparison matrix is incorrect and the normalized utilities are incorrect.

\subsubsection{AHP}

The \textbf{Analytic Hierarchy Process} was introduced in 1980 based on the following criticisms:
\begin{enumerate}
	\item The reconstruction of the single-variable normalized utility functions is subject to strong approximation errors
	
	\item The estimation of the weights is subject to strong approximation errors when the number of attributes $p$ is large
	
	\item The various approximation errors combine in cascade
\end{enumerate}

The method replaces absolute measures with relative ones, and quantitative ratios with qualitative scales, building a hierarchy of indicators to compare only conceptually similar quantities.

The preference among impacts is measured with an arbitrary qualitative scale, allowing to compare heterogeneous quantities, translating verbal judgments and building an evaluation matrix. 

Humans find it difficult to compare nonhomogeneous things, so build an indicator tree and compare only siblings: leaves include elementary attributes, upper levels summarize them, getting progressively more general.

% TODO Continue from page 96